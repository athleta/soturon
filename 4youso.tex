%!TEX root = 0卒業論文.tex
\newpage

\section{\rm 教材に組み込む要素}
本章では,教材を制作するにあたって,教材に組み込む要素をまとめた.



\subsection{小学校プログラミング教育への導入}
本研究で制作する教材は,小学校プログラミング教育への導入学習も出来るようにする.小学校プログラミング教育で目標としている要素である「知識・技能」,「思考力・判断力・表現力」,「学びに向かう力・人間性」より未就学児童向けとして最適化させるために「知識」「思考力」の2点に絞った.

\subsection{思考力}
思考力の要素よりプログラミング的思考力を身につけるとある.このプログラミング的思考力をより具体化すると以下の論理的思考力となる.これらの論理的思考力より,基本的なシーケンス,ループ,分岐,真偽値の存在を認知するすることを目標とする.
\begin{itemize}
\item 計算や作業を手順に分けて順序立てる「シーケンス」の考え方\\
\item 手順のまとまりを繰り返して実行する「ループ」の考え方\\
\item 条件によって作業を切り替える「分岐」の考え方\\
\item ものごとをYes/No の組み合わせで考える「真偽値」の考え方\\
\item ものごとの性質や手順のまとまりに名前をつける「抽象化」の考え方
\end{itemize}

\subsection{知識}
小学校教育ではこの「知識」要素より,身近な生活にコンピュータが活用されていることや,問題の解決に必要な手順があることに気づくことを目標にしている.未就学児向けに簡易化させ,プログラミングに触れることでプログラミングという概念の存在を認知することを目標とする.

\subsection{プログラミングの面白さを知る}
プログラミングを学習するにあたってプログラミングの面白さを知ることは,学習意欲に大きく影響する.具体的にプログラミングの面白さの要素とは何なのか以下にまとめる.
\begin{itemize}
\item ものを作ること\\
\item 他人の役に立つこと\\
\item 複雑な仕組みを見ること\\
\item 新しいものを知ること\\
\item 扱いやすい道具を使うこと
\end{itemize}
