%!TEX root = 0卒業論文.tex
\clearpage

\section{\rm 教材の設計}

\subsection{トップページ部}
トップページ部では9つのファイルで構成したサイトのトップページ部分である.
各ファイルの説明を表\ref{tab:topsetumei}に示す.
\begin{table}[htb]
\begin{center}
    \caption{トップページ部構成説明}
  \begin{tabular}{|c|c|} \hline
     ファイル名  & 説明  \\ \hline
     home.html& サイトのトップページとなる部分 \\ \hline
     site.html& サイトの利用方法や注意点を示す \\ \hline
     zyunnbann.html& 順番処理の問題選択画面 \\ \hline
     kurikaesi.html& 繰り返しの問題選択画面 \\ \hline
     zyoukennbunnki.html& 条件分岐の問題選択画面 \\ \hline
     message.html& メッセージの問題選択画面 \\ \hline
     ouyou.html& 応用問題の問題選択画面 \\ \hline
     scratch.html& Scratchの使用方法に関する部分 \\ \hline
     link.html& Scratch,文部科学省,学習指導要領のリンク\\ \hline
  \end{tabular}
  \label{tab:topsetumei}
  \end{center}
\end{table}
\newpage

\subsection{問題部}
問題部では制作した問題を出題する.

\subsubsection{順番処理}
順番処理の問題構成を表\ref{tab:zyunnbannhyou}に示す.
\begin{table}[htb]
\begin{center}
    \caption{順番処理構成説明}
  \begin{tabular}{|c|c|} \hline
     ファイル名  & 問題タイトル  \\ \hline
     monndai1.html& 猫を10歩動かし,大きさを変えてみよう! \\ \hline
     monndai2.html& 猫を元の位置、元の大きさに戻そう! \\ \hline
     monndai3.html& 猫を分身させながら歩かせてみよう! \\ \hline
     monndai4.html& 猫を喋らせてみよう! \\ \hline
     monndai5.html& 変数を使ってみよう! \\ \hline
     monndai6.html& 乱数を使ってみよう! \\ \hline
  \end{tabular}
  \label{tab:zyunnbannhyou}
  \end{center}
\end{table}

\subsubsection{繰り返し}
繰り返しの構成を表\ref{tab:kurikaesihyou}に示す.

\begin{table}[htb]
\begin{center}
    \caption{繰り返し構成説明}
  \begin{tabular}{|c|c|} \hline
     ファイル名  & 問題タイトル  \\ \hline
     monndai1.html& ループを使ってみよう! \\ \hline
     monndai2.html& 猫を鳴かせてみよう! \\ \hline
     monndai3.html& 猫をずっと歩かせてみよう! \\ \hline
     monndai4.html& 背景を変え続けてみよう! \\ \hline
     monndai5.html& 猫にアニメーションをつけてみよう! \\ \hline
     monndai6.html& 回数を指定したループを使ってみよう! \\ \hline
     monndai7.html& 時間を指定したループを使ってみよう! \\ \hline
     monndai8.html& 様々な条件をループに組み込もう! \\ \hline
     monndai9.html& ループの中にループを組み込んでみよう!1 \\ \hline
     monndai10.html& ループの中にループを組み込んでみよう!2 \\ \hline
     monndai11.html& ループの中にループを組み込んでみよう!3 \\ \hline
  \end{tabular}
  \label{tab:kurikaesihyou}
  \end{center}
\end{table}
\newpage

\subsubsection{条件分岐}
条件分岐処理の構成を表\ref{tab:zyoukenhyou}に示す.

\begin{table}[htb]
\begin{center}
    \caption{条件分岐構成説明}
  \begin{tabular}{|c|c|} \hline
     ファイル名  & 問題タイトル  \\ \hline
     monndai1.html& 足し算の答えを確認してみよう! \\ \hline
     monndai2.html& 特定の座標でのみ処理を行わせよう! \\ \hline
     monndai3.html& キーやマウスを条件に設定してみよう \\ \hline
     monndai4.html& 2つの条件下で動くプログラムを作ろう \\ \hline
     monndai5.html& 条件を満たしたときにループさせよう \\ \hline
     monndai6.html& 条件をさらに詳しく定義しよう \\ \hline
     monndai7.html& ループ中に条件分岐をさせてみよう \\ \hline
     monndai8.html& 1つのブロック2つの処理を用意する1 \\ \hline
     monndai9.html& 1つのブロック2つの処理を用意する2\\ \hline
  \end{tabular}
  \label{tab:zyoukenhyou}
  \end{center}
\end{table}

\subsubsection{応用問題}
応用問題の構成を表\ref{tab:ouyouhyou}
\begin{table}[htb]
\begin{center}
    \caption{応用問題構成説明}
  \begin{tabular}{|c|c|} \hline
     ファイル名  & 問題のねらい  \\ \hline
     monndai1.html& 今まで学習してきたことを組み合わせてプログラムを完成させる \\ \hline
     monndai2.html& 今まで学習してきたことを組み合わせてプログラムを完成させる \\ \hline
     monndai3.html& 今まで学習してきたことを組み合わせてプログラムを完成させる \\ \hline
     monndai4.html& 今まで学習してきたことを組み合わせて自分自身で創作物を完成させる \\ \hline
  \end{tabular}
  \label{tab:ouyouhyou}
  \end{center}
\end{table}


\subsection{解説部}
問題部下部に示す,「解説を見る」ボタンを押すことでモーダルウィンドウが開く.ウィンドウ内にて各問題ごとのスクラッチファイルのダウンロード,解答例,解説を表示する.












