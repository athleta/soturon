%!TEX root = 0卒業論文.tex
\newpage

\section{\rm 結言}

近年の情報化社会への急激な移り変わりや,IT技術者不足の背景などを受け,文部科学省が2020年以降に従来の教育にプログラミング教育を盛り込んだ学習指導要領改定案を発表した.
これにより小学校低学年時からプログラミング教育が実施されることになり,児童のプログラミング教育に注目が集まっている.

しかし,2020年に向けてプログラミングの授業に関する研修は各々行われているものの,授業を行えるか不安と感じる教職員が多い.


さらに,ビジュアルプログラミングの解説と例題の提供を同時に行なっているビジュアルプログラミング教材が不足しており,
ビジュアルプログラミングを学習した後,教職員は自分で授業用の例題を作成するか,別の教材や問題集などから授業で扱う例題などを引用してくる必要がある.


本研究では, 小学校プログラミング教育への導入, プログラミング的思考力を身につける, 教職員のプログラミング教育への知識不足による不安と負担を解消することのできるWebサイトを用いたプログラミング教材を制作した.

「Scratch」を用いて, 文部科学省が提唱した「プログラミング的思考」を構成する「自らの意図を明確化させる思考力」「どのような動きをどのような順序でさせれば良いのかを考える思考力」「どのように組み合わせれば良いかを考える思考力」を育成することを目的とした.
