%!TEX root = 0卒業論文.tex
\newpage

\section{\rm 緒言}
%背景
文部科学省が2020年以降に従来の教育にプログラミング教育を盛り込んだ学習指導要領改定案を発表した.これにより小学校低学年時からプログラミング教育が実施されることになり,児童のプログラミング教育に注目が集まっている.

%問題点
しかし,2020年に向けてプログラミングの授業に関する研修は各々行われているものの,授業を行えるか不安と感じる教職員が多い.実際に,東京都の教職員の85\%はプログラミング授業の実施経験がなく,98\%の教職員が授業の実施に不安を感じている[1].

さらに,ビジュアルプログラミングの解説と例題の提供を同時に行なっているビジュアルプログラミング教材が不足しており,ビジュアルプログラミングを学習した後,教職員は自分で授業用の例題を作成するか,別の教材や問題集などから授業で扱う例題などを引用してくる必要がある.

%目的
そこで本制作では,プログラミング経験及び授業経験のない小学校教員が児童に対して出題する例題を提供することと,それに対する教員の理解を促進することを目的としたプログラミング課題集Webサイトの制作を行うことを目的とする.
