%!TEX root = 0卒業論文.tex
\newpage

\addcontentsline{toc}{section}{参考文献}

\begin{thebibliography}{5}

\bibitem{yuushiki} 「小学校段階における論理的思考力や創造性、問題解決能力等の育成とプログラミング教育に関する有識者会議」,文部科学省, \url{http://www.mext.go.jp/b_menu/shingi/chousa/shotou/122/index.html}\\

\bibitem{scratch} 「Scratch」,MITメディアラボ ライフロングキンダーガーテングループ\url{https://scratch.mit.edu/}\\

\bibitem{viscuit} 「Viscuit」,合同会社デジタルポケット\url{https://www.viscuit.com/}\\

\bibitem{proguru}   「プログル」,みんなのコード,\url{https://proguru.jp/}\\

\bibitem{monkey} 「CodeMonkey」,J21 Corporation,\url{https://codemonkey.jp/}\\

\bibitem{lego} 「LEGO MINDSTORMS」,LEGO,\url{https://www.lego.com/ja-jp/products/themes/mindstorms}\\

\bibitem{blockly} 「Blockly」,Google Developers,\url{https://developers.google.com/blockly/}\\

\bibitem{ruby} リンダ,リウカス「ルビィのぼうけん:こんにちは!プログラミング」鳥井雪訳,翔泳社,2016年\\

\bibitem{con} 小林裕紀・兼宗進 「コンピュータを使わない小学校プログラミング教育:”ルビィのぼうけん”で育む論理的思考」翔泳社,2017年\\


\end{thebibliography}
