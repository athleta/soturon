%!TEX root = 0卒業論文.tex
\newpage

\section{\rm 小学校プログラミング必修化}
本章では,小学校プログラミング教育の必修化に至った背景や目標についてその説明する.

\subsection{必修化の背景}
小学校プログラミング教育の必修化の背景として,以下に文部科学省主催の有識者会議でまとめられたものから抜粋する.


\begin{itemize}
 \item 近年,飛躍的に発展した人工知能に対し,人間はみずみすしい感性を働かせながら創造的な問題解
決を行うことができる強みを持っている.こうした人間の強みを伸ばしていくものが長年の学校
教育の長年の目標である\\

 \item 自動販売機やロボット掃除機など,身近な生活の中でもコンピュータとプログラミングの働きの
恩恵を受けており,これらの物がプログラミングを通じて人間の意図した処理を行わせることが
できるものであることを理解できるようにする必要がある\\

 \item 小学校段階におけるプログラミング教育については,コーディングではなくコンピュータに意図
した処理を行うように指示することができるということを体験させながら,将来どのような職業
に就くとしても時代を超えて普遍的に求められる力としての「プログラミング的思考」などを育成
する
\end{itemize}

\subsection{プログラミング的思考とは}
プログラミング的思考とは,以下に文部科学省主催の有識者会議でまとめられたものから抜粋する.
\\
\\
``自分が意図する一連の活動を実現するために,どのような動きの組み合わせが必要であり,1 つ1 つの
動きに対応した記号を,どのように組み合わせたらいいのか,記号の組み合わせをどのように改善して
いけば,より意図した活動に近づくのかということを論理的に考えていく力である.''



\subsection{プログラミング教育を通じて目指す育成すべき資質・能力}
プログラミング教育を通じて目指す育成すべき資質・能力について,以下に文部科学省主催の有識者会議でまとめられたものから抜粋する.\\
\\
【知識・技能】\\
身近な生活でコンピュータが活用されていることや,問題の解決には必要な手順があることに気づく
こと.\\
\\
【思考力・判断力・表現力】\\
発展の段階に即して,「プログラミング的思考」を育成すること.\\
\\
【学びに向かう力・人間性】\\
発達の段階に即して,コンピュータの動きを,よりよい人生や社会づくりに生かそうとする態度を涵
養すること.\\
\\

\subsection{小学校段階におけるプログラミング教育の実用例}
小学校で必修化されるプログラミング教育は,教科化ではない.そのため算数や国語のような教科ごとの時間は用意されない.そのため既存の教科にプログラミング要素を関連付けて教育することになる.以下に文部科学省主催の有識者会議でまとめられた実用例を抜粋する.\\\\

\begin{table}[htb]
    \caption{小学校段階におけるプログラミング教育の実用例}
  \begin{tabularx}{\linewidth}{|X|X|} \hline
    教科& 実用例 \\ \hline
    総合学習& 自分の暮らしとプログラミングとの関係を考え,そのよさに気づく学習 \\ \hline
    理科&電気製品にはプログラムが活用され条件に応じて動作していることに気づく学習\\ \hline
    算数 & 図の作成において,プログラミング的思考と数学的な思考の関係やよさに気づく学習\\ \hline
    音楽 & 創作用のICT ツールを活用しながら,音の長さや高さの組み合わせなど試行錯誤し,音楽をつくる学習\\ \hline
    図画工作 & 表現しているものを,プログラミングを通じて動かすことにより,新たな発送や構想を生み出す学習\\ \hline
   
  \end{tabularx}

  \end{table}
\newpage


\subsubsection{コンピュータを使わない授業の実践例}
小学校プログラミング必修化において実際に行われた実験例\cite{con}を以下にまとめる.\\
\\
\begin{table}[htb]
\begin{center}
\centering
    \caption{コンピュータを使わない授業の実践例}
  \begin{tabular}{|c|c|l|} \hline
    教科& 学年&実践例 \\ \hline
    国語& 3&文章の構成をシーケンスの考え方を用いて考える授業\\ \hline
    算数& 3&筆算の仕方をシーケンスの考え方を用いて考える授業\\ \hline
    理科& 3&身の回りの物を条件でグループ分けする授業\\ \hline
    音楽& 3&曲に合わせた三拍子のリズムをループなどを用いて構築する授業\\ \hline
    算数& 4&ベン図を用いた条件分けの授業\\ \hline
    学級活動& 4&仲間の個性を真偽クイズとして作成し,当て合うレクリエーション\\ \hline
    外国語活動& 6&マス目わけした地図を英語で指示をして最短経路で目的地まで進む\\ \hline
   
  \end{tabular}
  \label{tab:bamen1}
  \end{center}
\end{table}

\subsection{考察}
小学校プログラミング教育への導入を目指すため有識者会議でまとめられた要素を取り組む必要がある.実践例,実用例よりシーケンスやループなど論理的思考力を高めるうまく既存の教材に組み込んで教えているのがわかる.同じように未就学児でも未就学児童に身近な存在である紙芝居や絵本などに組み込むのが良いと考える.




%【参考文献(産経WEST)など】
